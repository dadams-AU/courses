\documentclass[10pt]{beamer}
\usetheme{metropolis}
\usepackage{booktabs}
\usepackage{tabularx}
\usepackage{calc}
\usepackage{tikz}

% Setup for faculty images
\newlength{\imageheight}
\setlength{\imageheight}{3.5cm}

% Define CSUF brand colors
\definecolor{titanblue}{HTML}{00244E}
\definecolor{mediumblue}{HTML}{0F3F8C}
\definecolor{skyblue}{HTML}{EBFBFF}
\definecolor{titanorange}{HTML}{FF7900}
\definecolor{titangray}{HTML}{F5F5F5}
\definecolor{titantext}{HTML}{222222}

% Customize metropolis theme colors
\setbeamercolor{normal text}{fg=titantext, bg=white}
\setbeamercolor{alerted text}{fg=titanorange}
\setbeamercolor{example text}{fg=mediumblue}

% Title page colors
\setbeamercolor{title}{fg=titanblue, bg=white}
\setbeamercolor{subtitle}{fg=mediumblue, bg=white}
\setbeamercolor{institute}{fg=titanorange, bg=white}
\setbeamercolor{date}{fg=titanblue, bg=white}

% Frame title colors
\setbeamercolor{frametitle}{fg=white, bg=titanblue}
\setbeamercolor{framesubtitle}{fg=mediumblue, bg=white}

% Block environment colors
\setbeamercolor{block title}{fg=white, bg=titanblue}
\setbeamercolor{block body}{fg=titantext, bg=skyblue!10}

% Item colors
\setbeamercolor{itemize item}{fg=titanorange}
\setbeamercolor{itemize subitem}{fg=mediumblue}
\setbeamercolor{itemize subsubitem}{fg=titanblue}

% Footer and header colors
\setbeamercolor{footer}{fg=titantext}
\setbeamercolor{header}{fg=titanblue}

% Customize fonts
\setbeamerfont{title}{size=\Large, series=\bfseries}
\setbeamerfont{frametitle}{size=\large, series=\bfseries}

% Simple title page template
\defbeamertemplate*{title page}{customized}[1][]
{
\vspace{1cm}
 {\usebeamerfont{title}\usebeamercolor[fg]{title}\inserttitle\par}
\vspace{0.5cm}
 {\usebeamerfont{subtitle}\usebeamercolor[fg]{subtitle}\insertsubtitle\par}
\vspace{0.5cm}
 {\usebeamerfont{date}\usebeamercolor[fg]{date}\insertdate\par}
\vfill
 {\insertinstitute\par}
}

% Add progress bar
\makeatletter
\setbeamertemplate{headline}{%
\begin{beamercolorbox}[wd=\paperwidth,ht=0.4cm,dp=0cm]{titanblue}%
\begin{tikzpicture}
\fill[titanorange] (0,0) rectangle (\the\paperwidth*\insertframenumber/\inserttotalframenumber,0.4cm);
\end{tikzpicture}%
\end{beamercolorbox}%
}
\makeatother

\begin{document}

\title{Understanding Politics and Public Policy}
\subtitle{Foundations and Core Concepts\\POSC 315: Introduction to Public Policy\\Lecture 1 (Part 1 of 3)}
\date{}
\institute{California State University, Fullerton}

\maketitle

% Course Theme
\begin{frame}
\frametitle{Theme of the Course}
\framesubtitle{American Political Values}

\begin{block}{}
\begin{enumerate}
\item Individualism
\item Equality
\item Community
\item Patriotism
\item Rule of Law
\item Diversity
\item Distrust of Government
\end{enumerate}
\end{block}

\end{frame}

% Opening Questions
\begin{frame}
\frametitle{Let's start with a couple of questions:}

\begin{columns}
\begin{column}{0.48\textwidth}
\begin{block}{}
\centering
\textcolor{mediumblue}{\textbf{Why do we have programs for reduced cost or free school lunches?}}
\end{block}
\end{column}

\begin{column}{0.48\textwidth}
\begin{block}{}
\centering
\textcolor{titanblue}{\textbf{Why is the primary responsibility for educating children and policing people left to state and local governments?}}
\end{block}
\end{column}
\end{columns}

\vspace{1cm}
\pause
\centering
These questions help us understand \textbf{values} and \textbf{priorities} in policy choices.

\end{frame}

% Introduction
\begin{frame}
\frametitle{Introduction}

\begin{itemize}
\item<1-> We seek to understand and find solutions for public problems.

\item<2-> We have many theories about how the policy process works.
\begin{itemize}
\item Many are \emph{interdisciplinary}.
\end{itemize}

\item<3-> We will focus on the \emph{politics} of policy.
\begin{itemize}
\item How do we get from a problem to a solution?
\item How do we get from a solution to a policy?
\item How do we get from a policy to a program?
\item How do we get from a program to an outcome?
\end{itemize}
\end{itemize}

\end{frame}

% Different Views
\begin{frame}
\frametitle{Different Views Across the Political Landscape}

\begin{alertblock}{}
Across the political landscape, we have many different views about:
\end{alertblock}

\vspace{0.5cm}

\begin{columns}
\begin{column}{0.48\textwidth}
\begin{block}{1. Problem Definition}
What things are problems that need solving?
\end{block}

\vspace{0.3cm}

\begin{block}{3. Intervention Selection}
Is a government program the best way to solve those problems?
\end{block}
\end{column}

\begin{column}{0.48\textwidth}
\begin{block}{2. Solution Identification}
What are the solutions to those problems?
\end{block}

\vspace{0.3cm}

\begin{block}{4. Implementation Strategy}
What are the best ways to implement solutions?
\end{block}
\end{column}
\end{columns}

\end{frame}

% Defining Politics - Question
\begin{frame}
\frametitle{Politics and Public Policy Definitions}

\begin{block}{}
\centering
\textcolor{mediumblue}{\textbf{How did your friend, family member, or other person respond when you asked them to define politics?}}

\vspace{0.5cm}

\textcolor{titanblue}{\textbf{What about policy?}}
\end{block}

\vspace{1cm}
\pause
\centering
Let's compare your responses with academic definitions...

\end{frame}

% Politics Definition
\begin{frame}
\frametitle{Politics and the Policy Process}

\begin{block}{Politics}
\begin{itemize}
\item<1-> The process of making \textbf{collective decisions}, usually by governments, to allocate public resources and to create and enforce rules for the operation of society.

\item<2-> How we organize and govern ourselves; the \textbf{art and science of government}.
\end{itemize}
\end{block}

\end{frame}

% Public Policy Definition
\begin{frame}
\frametitle{Politics and the Policy Process}

\begin{block}{Public Policy}
\begin{itemize}
\item<1-> The \textbf{course of action} the government takes in response to an issue or problem.

\item<2-> Public policy is political because it takes place in the \emph{public sphere}.

\item<3-> Public policy addresses problems that are public or problems that some members of society think \emph{should} be public.
\end{itemize}
\end{block}

\end{frame}

% What is Public?
\begin{frame}
\frametitle{What is Public?}

\pause
\begin{itemize}
\item Public versus Private come to us from the Latin \emph{publicus} and \emph{privatus}, from Ancient Rome
\item Publicus means ``of the people'' or ``of the state''
\item Privatus means ``individual'' or ``personal''
\end{itemize}

\end{frame}

% Public vs Private Table
\begin{frame}
\frametitle{What is Public?}

\begin{table}
\centering
\begin{tabular}{|l|l|}
\hline
\textbf{Public (publicus)} & \textbf{Private (privatus)} \\
\hline
\pause
Polis -- the State & The Household -- private business \\
\hline
\pause
Freedom & Necessity \\
\hline
\pause
Equality & Inequality \\
\hline
\pause
Immortality & Mortality \\
\hline
\pause
Open & Closed \\
\hline
\end{tabular}
\end{table}

\vspace{1cm}
\pause
\centering
These distinctions begin to collapse from the 19th century onward.

\end{frame}

% Reflection Activity
\begin{frame}
\frametitle{Reflection Activity}
\framesubtitle{Consider how the boundaries between ``public'' and ``private'' have blurred in modern society.}

\begin{columns}
\begin{column}{0.48\textwidth}
\begin{block}{Think About:}
\begin{itemize}
\item Social media platforms
\item Healthcare decisions
\item Education choices
\end{itemize}
\end{block}
\end{column}

\begin{column}{0.48\textwidth}
\begin{block}{Questions:}
\begin{itemize}
\item Where do you see the public/private boundary shifting?
\item How might these shifts affect policy decisions?
\end{itemize}
\end{block}
\end{column}
\end{columns}

\end{frame}

% Conclusion
\begin{frame}
\frametitle{Key Takeaways}

\begin{block}{}
\begin{itemize}
\item \textbf{Politics:} The process of making collective decisions about public resources and rules
\item \textbf{Public Policy:} Government's course of action in response to public issues
\item \textbf{Public vs. Private:} Historically distinct spheres that are increasingly blurred
\item \textbf{Policy Process:} The path from problem identification to outcome evaluation
\end{itemize}
\end{block}

\pause
\vspace{1cm}
\begin{center}
\textbf{Coming Up Next}

In Part 2, we'll explore the history of political thought and how it shapes modern policy approaches.
\end{center}

\end{frame}

\end{document}