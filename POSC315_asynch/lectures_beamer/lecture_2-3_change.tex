\documentclass[10pt]{beamer}
\usetheme{metropolis}
\usepackage{tikz}

% Define CSUF brand colors
\definecolor{titanblue}{HTML}{00244E}
\definecolor{mediumblue}{HTML}{0F3F8C}
\definecolor{titanorange}{HTML}{FF7900}
\definecolor{titantext}{HTML}{222222}

% Additional colors for theories
\definecolor{streamblue}{HTML}{2563EB}
\definecolor{streamgreen}{HTML}{059669}
\definecolor{streampurple}{HTML}{7C3AED}
\definecolor{coalitionred}{HTML}{DC2626}
\definecolor{punctuationorange}{HTML}{EA580C}
\definecolor{equilibriumgreen}{HTML}{059669}

% Customize colors
\setbeamercolor{frametitle}{fg=white, bg=titanblue}
\setbeamercolor{block title}{fg=white, bg=titanblue}
\setbeamercolor{block title alerted}{fg=white, bg=punctuationorange}
\setbeamercolor{itemize item}{fg=titanorange}

\begin{document}

\title{Understanding Policy Change}
\subtitle{Introduction to Policy Process Theories}
\author{David P. Adams, Ph.D.}
\institute{California State University, Fullerton}
\date{Summer 2025}

\maketitle

\begin{frame}
\frametitle{What Is Policy Change?}

Policy change refers to significant shifts in government action, law, or regulation over time.

\vspace{0.5cm}

\begin{itemize}
  \item Why do some policies stay the same for decades, while others change rapidly?
  \item What triggers major reforms or reversals?
  \item How do ideas, interests, and institutions interact to shape outcomes?
\end{itemize}

\vspace{0.5cm}

\begin{alertblock}{Today's Goal}
Introduce the main theories that help us answer these questions.
\end{alertblock}

\end{frame}

\begin{frame}
\frametitle{Why Study Policy Process Theories?}

Theories help us:

\begin{itemize}
  \item \textbf{Understand Change}: Make sense of how and why policies evolve over time
  \item \textbf{Identify Key Factors}: Recognize the crucial elements that drive policy development  
  \item \textbf{Predict Change}: Anticipate when conditions are ripe for policy shifts
  \item \textbf{Analyze History}: Examine past policy successes and failures through theoretical lenses
\end{itemize}

\end{frame}

\begin{frame}
\frametitle{Three Major Theoretical Frameworks}

\begin{block}{\textcolor{streamblue}{Multiple Streams Framework}}
Policy windows open when problem, policy, and politics streams converge
\end{block}

\begin{block}{\textcolor{coalitionred}{Advocacy Coalition Framework}}
Policy change through competing coalitions with shared belief systems
\end{block}

\begin{block}{\textcolor{punctuationorange}{Punctuated Equilibrium Theory}}
Long periods of stability interrupted by sudden, dramatic shifts
\end{block}

\end{frame}

\begin{frame}
\frametitle{Kingdon's Multiple Streams Framework}

Policy windows open when three independent streams converge:

\vspace{0.5cm}

\begin{block}{\textcolor{streamblue}{1. Problem Stream}}
Issues gaining attention

Example: Opioid crisis or rising homelessness
\end{block}

\begin{block}{\textcolor{streamgreen}{2. Policy Stream}}
Available solutions

Example: Harm reduction or housing-first policies
\end{block}

\begin{block}{\textcolor{streampurple}{3. Politics Stream}}
Political conditions

Example: Bipartisan support or public demand
\end{block}

\end{frame}

\begin{frame}
\frametitle{Multiple Streams: Key Concepts}

\begin{itemize}
  \item \textbf{Policy Entrepreneurs}: Individuals who promote solutions and connect streams
  \item \textbf{Policy Windows}: Brief opportunities for policy change  
  \item \textbf{Coupling}: The process of linking problems to solutions in the right political moment
\end{itemize}

\vspace{0.5cm}

\begin{alertblock}{Real-World Example}
The Clean Air Act amendments of 1990, where environmental concerns, available policy solutions, and bipartisan political support converged at the right moment.
\end{alertblock}

\end{frame}

\begin{frame}
\frametitle{Advocacy Coalition Framework}

Policy change occurs through competing coalitions organized around shared beliefs.

\vspace{0.5cm}

\begin{block}{1. Shared Beliefs}
Groups form around common values and policy goals
\end{block}

\begin{block}{2. Competition}
Coalitions compete to influence policy decisions
\end{block}

\begin{block}{3. Adaptation}
Coalitions learn and adjust strategies over time
\end{block}

\end{frame}

\begin{frame}
\frametitle{Advocacy Coalition: Key Concepts}

\begin{itemize}
  \item \textbf{Belief Systems}: Core values that unite coalition members
  \item \textbf{Policy Subsystems}: Specific policy areas where coalitions compete
  \item \textbf{Policy Learning}: How coalitions adapt their strategies based on experience
\end{itemize}

\vspace{0.5cm}

\begin{alertblock}{Real-World Example}
The long-term debate between environmental advocates and the fossil fuel industry over climate policy, with each coalition learning and adapting strategies over decades.
\end{alertblock}

\end{frame}

\begin{frame}
\frametitle{Punctuated Equilibrium Theory}

Policy changes through long periods of stability interrupted by sudden, dramatic shifts.

\vspace{0.5cm}

\begin{block}{\textcolor{equilibriumgreen}{1. Stability}}
Long periods of incremental change or no change
\end{block}

\begin{block}{\textcolor{punctuationorange}{2. Punctuation}}
Sudden, dramatic policy shifts
\end{block}

\begin{block}{\textcolor{equilibriumgreen}{3. New Equilibrium}}
Return to stability at a new policy position
\end{block}

\end{frame}

\begin{frame}
\frametitle{Punctuated Equilibrium: Key Concepts}

\begin{itemize}
  \item \textbf{Policy Images}: How issues are understood and framed
  \item \textbf{Venue Shopping}: Moving issues to favorable decision-making venues
  \item \textbf{Attention Shifts}: Rapid changes in focus after long periods of inattention
\end{itemize}

\vspace{0.5cm}

\begin{alertblock}{Real-World Example}
Major civil rights legislation in the 1960s, which marked a sudden shift after years of incremental or no change, establishing a new policy equilibrium.
\end{alertblock}

\end{frame}

\begin{frame}
\frametitle{Comparing the Theories}

\textbf{Multiple Streams Framework}
\begin{itemize}
\item Focus: Timing and opportunity
\item Change Mechanism: Convergence of streams
\item Key Insight: Windows of opportunity are rare and brief
\end{itemize}

\textbf{Advocacy Coalition Framework}
\begin{itemize}
\item Focus: Group dynamics
\item Change Mechanism: Coalition competition
\item Key Insight: Beliefs drive policy positions
\end{itemize}

\textbf{Punctuated Equilibrium Theory}
\begin{itemize}
\item Focus: Patterns over time
\item Change Mechanism: Rapid shifts after stability
\item Key Insight: Change is often episodic rather than gradual
\end{itemize}

\end{frame}

\begin{frame}
\frametitle{Applying Theory: Healthcare Reform}

\textbf{\textcolor{streamblue}{Multiple Streams}}
\begin{itemize}
\item Problem: Rising costs
\item Policy: Different reform models
\item Politics: Party control of government
\end{itemize}

\textbf{\textcolor{coalitionred}{Advocacy Coalition}}
\begin{itemize}
\item Progressive reform coalition
\item Market-based reform coalition
\item Status quo coalition
\end{itemize}

\textbf{\textcolor{punctuationorange}{Punctuated Equilibrium}}
\begin{itemize}
\item Long periods of debate
\item Sudden passage of legislation
\item New implementation equilibrium
\end{itemize}

\end{frame}

\begin{frame}
\frametitle{Summary}

\textbf{Key Takeaways:}
\begin{itemize}
  \item Policy process theories help us understand and predict policy changes
  \item \textbf{Kingdon's Multiple Streams Framework} highlights the convergence of problem, policy, and politics streams
  \item The \textbf{Advocacy Coalition Framework} focuses on belief systems and coalition dynamics
  \item \textbf{Punctuated Equilibrium Theory} explains long periods of stability interrupted by sudden changes
\end{itemize}

\vspace{1cm}

Different theories highlight different aspects of the same policy story.

\end{frame}

\end{document}