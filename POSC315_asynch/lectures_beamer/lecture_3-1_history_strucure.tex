\documentclass[10pt]{beamer}
\usetheme{metropolis}
\usepackage{booktabs}
\usepackage{tabularx}
\usepackage{calc}
\usepackage{tikz}
\usepackage{fontawesome5}

% Setup for faculty images
\newlength{\imageheight}
\setlength{\imageheight}{3.5cm}

% Define CSUF brand colors
\definecolor{titanblue}{HTML}{00244E}
\definecolor{mediumblue}{HTML}{0F3F8C}
\definecolor{skyblue}{HTML}{EBFBFF}
\definecolor{titanorange}{HTML}{FF7900}
\definecolor{titangray}{HTML}{F5F5F5}
\definecolor{titantext}{HTML}{222222}

% Additional colors for variety
\definecolor{successcolor}{HTML}{27AE60}
\definecolor{infocolor}{HTML}{3498DB}
\definecolor{accentcolor}{HTML}{9B59B6}

% Customize metropolis theme colors
\setbeamercolor{normal text}{fg=titantext, bg=white}
\setbeamercolor{alerted text}{fg=titanorange}
\setbeamercolor{example text}{fg=mediumblue}

% Title page colors
\setbeamercolor{title}{fg=titanblue, bg=white}
\setbeamercolor{subtitle}{fg=mediumblue, bg=white}
\setbeamercolor{institute}{fg=titanorange, bg=white}
\setbeamercolor{date}{fg=titanblue, bg=white}

% Frame title colors
\setbeamercolor{frametitle}{fg=white, bg=titanblue}
\setbeamercolor{framesubtitle}{fg=mediumblue, bg=white}

% Block environment colors
\setbeamercolor{block title}{fg=white, bg=titanblue}
\setbeamercolor{block body}{fg=titantext, bg=skyblue!10}

% Example block colors
\setbeamercolor{block title example}{fg=white, bg=successcolor}
\setbeamercolor{block body example}{fg=titantext, bg=successcolor!10}

% Alert block colors
\setbeamercolor{block title alerted}{fg=white, bg=infocolor}
\setbeamercolor{block body alerted}{fg=titantext, bg=infocolor!10}

% Item colors
\setbeamercolor{itemize item}{fg=titanorange}
\setbeamercolor{itemize subitem}{fg=mediumblue}
\setbeamercolor{itemize subsubitem}{fg=titanblue}

% Footer and header colors
\setbeamercolor{footer}{fg=titantext}
\setbeamercolor{header}{fg=titanblue}

% Customize fonts
\setbeamerfont{title}{size=\Large, series=\bfseries}
\setbeamerfont{frametitle}{size=\large, series=\bfseries}

% Simple title page template
\defbeamertemplate*{title page}{customized}[1][]
{
\vspace{1cm}
 {\usebeamerfont{title}\usebeamercolor[fg]{title}\inserttitle\par}
\vspace{0.5cm}
 {\usebeamerfont{subtitle}\usebeamercolor[fg]{subtitle}\insertsubtitle\par}
\vspace{0.5cm}
 {\usebeamerfont{date}\usebeamercolor[fg]{date}\insertdate\par}
\vfill
 {\insertinstitute\par}
}

% Add progress bar
\makeatletter
\setbeamertemplate{headline}{%
\begin{beamercolorbox}[wd=\paperwidth,ht=0.4cm,dp=0cm]{titanblue}%
\begin{tikzpicture}
\fill[titanorange] (0,0) rectangle (\the\paperwidth*\insertframenumber/\inserttotalframenumber,0.4cm);
\end{tikzpicture}%
\end{beamercolorbox}%
}
\makeatother

\begin{document}

\title{Historical and Structural Contexts}
\subtitle{The Foundation of the Policy Process\\POSC 315: Introduction to Public Policy\\Lecture 3.1}
\date{Summer 2025}
\institute{California State University, Fullerton}

\maketitle

% Why Study Policy Structures?
\begin{frame}
\frametitle{Why Study Policy Structures?}

\begin{block}{}
\textbf{Why does it take so long to make or change policy?}

Why do some big problems stick around for decades? Today we look at the structures that shape---and often slow down---public policy in the U.S.
\end{block}

\end{frame}

% Institutions: The Rules of the Game
\begin{frame}
\frametitle{Institutions: The Rules of the Game}

\begin{block}{}
Institutions are the \textbf{rules and organizations} that determine who gets to decide, what's allowed, and what's not.

They include laws, agencies, and even unwritten norms that shape our lives.
\end{block}

\end{frame}

% Examples: Rules & Institutions
\begin{frame}
\frametitle{Rules \& Institutions: Examples}

\begin{columns}
\begin{column}{0.48\textwidth}
\begin{block}{Rules}
\pause
\begin{itemize}
\item Constitution
\item Laws
\item Regulations
\item Norms
\end{itemize}
\end{block}
\end{column}

\begin{column}{0.48\textwidth}
\begin{block}{Institutions}
\pause
\begin{itemize}
\item Supreme Court
\item Congress
\item Executive Branch
\item Civil Society
\end{itemize}
\end{block}
\end{column}
\end{columns}

\end{frame}

% Formal vs Informal Institutions
\begin{frame}
\frametitle{Types of Institutions}

\begin{columns}
\begin{column}{0.48\textwidth}
\begin{block}{Formal Institutions}
\pause
\begin{itemize}
\item Government Agencies
\item Courts
\item Schools
\item Police
\item Corporations
\end{itemize}
\end{block}
\end{column}

\begin{column}{0.48\textwidth}
\begin{block}{Informal Institutions}
\pause
\begin{itemize}
\item Family
\item Religion
\item Media
\item Traditions
\item Social Norms
\end{itemize}
\end{block}
\end{column}
\end{columns}

\end{frame}

% The Constitution: A Living Document
\begin{frame}
\frametitle{The Constitution: A Living Document}

\begin{itemize}
\item<1-> \textbf{Vague by design}: open to interpretation.
\item<2-> \textbf{Elasticity}: allows adaptation over time.
\item<3-> \textbf{Longevity}: still the foundation after 200+ years.
\end{itemize}

\pause
\vspace{1cm}
\begin{alertblock}{Think about it:}
Why did the framers make it so flexible?
\end{alertblock}

\end{frame}

% Separation of Powers
\begin{frame}
\frametitle{Separation of Powers}

\begin{alertblock}{Key Concept}
\textbf{Separation of Powers} divides government into branches to prevent one group from taking too much control.
\end{alertblock}

\pause
\vspace{0.5cm}

\begin{exampleblock}{Checks \& Balances}
Each branch can limit the powers of the others. No branch rules alone.
\end{exampleblock}

\pause
\vspace{0.5cm}

\centering
\emph{What happens when these checks break down?} \pause \emph{You get gridlock---or sometimes, runaway power.}

\end{frame}

% Branch: Legislative
\begin{frame}
\frametitle{Branch: Legislative}

\pause
\begin{block}{\textcolor{titanblue}{Legislative}}
\begin{itemize}
\item Makes laws
\item Controls budget
\item Declares war
\end{itemize}
\end{block}

\end{frame}

% Branch: Executive
\begin{frame}
\frametitle{Branch: Executive}

\pause
\begin{block}{\textcolor{mediumblue}{Executive}}
\begin{itemize}
\item Enforces laws
\item Commander-in-Chief
\item Handles foreign policy
\end{itemize}
\end{block}

\end{frame}

% Branch: Judicial
\begin{frame}
\frametitle{Branch: Judicial}

\pause
\begin{block}{\textcolor{accentcolor}{Judicial}}
\begin{itemize}
\item Interprets laws
\item Reviews cases
\item Checks constitutionality
\end{itemize}
\end{block}

\end{frame}

% Where Policy Gets Made: Key Powers
\begin{frame}
\frametitle{Where Policy Gets Made: Key Powers}

\begin{columns}
\begin{column}{0.48\textwidth}
\begin{block}{Article I, Section 8}
\pause
\begin{itemize}
\item Taxation
\item Commerce
\item Defense
\item Naturalization
\item Intellectual Property
\end{itemize}

\vspace{0.3cm}
\textcolor{titanblue}{\textbf{Necessary \& Proper Clause}}
\end{block}
\end{column}

\begin{column}{0.48\textwidth}
\begin{block}{Amendments}
\pause
\begin{itemize}
\item 10th: Powers reserved to states
\item 14th: Equal protection, due process
\end{itemize}

\vspace{0.3cm}
\textcolor{mediumblue}{\textbf{Who decides? Sometimes: the courts.}}
\end{block}
\end{column}
\end{columns}

\end{frame}

% Policy Restraint: Why the System is Slow
\begin{frame}
\frametitle{Policy Restraint: Why the System is Slow}

\begin{block}{}
The U.S. system is full of \textbf{roadblocks}---and that's on purpose. Federalism, separation of powers, bicameralism, judicial review, and the amendment process all slow things down.
\end{block}

\vspace{0.5cm}

\begin{columns}
\begin{column}{0.32\textwidth}
\pause
\begin{center}
\textcolor{titanblue}{\textbf{Federalism}}
\end{center}

\pause
\begin{center}
\textcolor{titanblue}{\textbf{Judicial Review}}
\end{center}
\end{column}

\begin{column}{0.32\textwidth}
\pause
\begin{center}
\textcolor{titanblue}{\textbf{Bicameralism}}
\end{center}

\pause
\begin{center}
\textcolor{titanblue}{\textbf{Checks \& Balances}}
\end{center}
\end{column}

\begin{column}{0.32\textwidth}
\pause
\begin{center}
\textcolor{titanblue}{\textbf{Amendments}}
\end{center}

\pause
\begin{center}
\textcolor{titanblue}{\textbf{Separation of Powers}}
\end{center}
\end{column}
\end{columns}

\end{frame}

% Incrementalism: Change by Inches
\begin{frame}
\frametitle{Incrementalism: Change by Inches}

\begin{alertblock}{Definition}
Most policy change is gradual---think evolution, not revolution.
\end{alertblock}

\vspace{0.5cm}

\begin{itemize}
\item<1-> \textbf{Minor Changes}: Small tweaks to existing policies
\item<2-> \textbf{Extensions}: Extending policies to new groups  
\item<3-> \textbf{Modifications}: Adjusting for new circumstances
\end{itemize}

\pause
\vspace{1cm}

\centering
\emph{Example: Legalizing marijuana---state by state, year by year.}

\end{frame}

% Big Takeaways
\begin{frame}
\frametitle{Big Takeaways}

\begin{itemize}
\item<1-> Policy change is slow---by design.
\item<2-> Institutions structure what's possible and what's not.
\item<3-> Most change is incremental, not sweeping.
\item<4-> Crises can sometimes speed things up.
\end{itemize}

\end{frame}

% Discussion
\begin{frame}
\frametitle{Discussion}

\begin{block}{}
Where have you seen policy structures slow down or speed up change?

Can you think of a recent policy change that happened quickly? What made it possible?
\end{block}

\end{frame}

\end{document}