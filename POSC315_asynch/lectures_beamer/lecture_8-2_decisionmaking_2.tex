\documentclass[10pt]{beamer}
\usetheme{metropolis}
\usepackage{booktabs}
\usepackage{tabularx}
\usepackage{calc}
\usepackage{tikz}
\usetikzlibrary{shapes.geometric, arrows, positioning, decorations.pathreplacing, patterns}
\usepackage[sfdefault]{FiraSans}
\usepackage[scaled]{FiraMono}

% Setup for faculty images
\newlength{\imageheight}
\setlength{\imageheight}{3.5cm}

% Define CSUF brand colors
\definecolor{titanblue}{HTML}{00244E}
\definecolor{mediumblue}{HTML}{0F3F8C}
\definecolor{skyblue}{HTML}{EBFBFF}
\definecolor{titanorange}{HTML}{FF7900}
\definecolor{titangray}{HTML}{F5F5F5}
\definecolor{titantext}{HTML}{222222}

% Customize metropolis theme colors
\setbeamercolor{normal text}{fg=titantext, bg=white}
\setbeamercolor{alerted text}{fg=titanorange}
\setbeamercolor{example text}{fg=mediumblue}

% Title page colors
\setbeamercolor{title}{fg=titanblue, bg=white}
\setbeamercolor{subtitle}{fg=mediumblue, bg=white}
\setbeamercolor{institute}{fg=titanorange, bg=white}
\setbeamercolor{date}{fg=titanblue, bg=white}

% Frame title colors
\setbeamercolor{frametitle}{fg=white, bg=titanblue}
\setbeamercolor{framesubtitle}{fg=mediumblue, bg=white}

% Block environment colors
\setbeamercolor{block title}{fg=white, bg=titanblue}
\setbeamercolor{block body}{fg=titantext, bg=skyblue!10}

% Item colors
\setbeamercolor{itemize item}{fg=titanorange}
\setbeamercolor{itemize subitem}{fg=mediumblue}
\setbeamercolor{itemize subsubitem}{fg=titanblue}

% Footer and header colors
\setbeamercolor{footer}{fg=titantext}
\setbeamercolor{header}{fg=titanblue}

% Customize fonts
\setbeamerfont{title}{size=\Large, series=\bfseries}
\setbeamerfont{frametitle}{size=\large, series=\bfseries}

% Simple title page template
\defbeamertemplate*{title page}{customized}[1][]
{
\vspace{1cm}
 {\usebeamerfont{title}\usebeamercolor[fg]{title}\inserttitle\par}
\vspace{0.5cm}
 {\usebeamerfont{subtitle}\usebeamercolor[fg]{subtitle}\insertsubtitle\par}
\vspace{0.5cm}
 {\usebeamerfont{date}\usebeamercolor[fg]{date}\insertdate\par}
\vfill
 {\insertinstitute\par}
}

% Add progress bar - FIXED VERSION
\makeatletter
\setbeamertemplate{headline}{%
\begin{beamercolorbox}[wd=\paperwidth,ht=0.4cm,dp=0cm]{titanblue}%
\begin{tikzpicture}
\pgfmathsetmacro{\progress}{\insertframenumber/\inserttotalframenumber}
\fill[titanorange] (0,0) rectangle (\progress*\paperwidth,0.4cm);
\end{tikzpicture}%
\end{beamercolorbox}%
}
\makeatother

\begin{document}

\title{Understanding Politics and Public Policy}
\subtitle{Foundations and Core Concepts\\POSC 315: Introduction to Public Policy\\Lecture 8-2\\Decision Making (Part 2 of 3)}
\author{Dr. David P. Adams}
\date{Summer 2025}
\institute{California State University, Fullerton}

\maketitle

% --- BEGIN SLIDES CONTENT ---

% Core Concepts in Bounded Rationality
\begin{frame}{Core Concepts in Bounded Rationality}
\begin{columns}
\begin{column}{0.5\textwidth}
\begin{block}{Intertwined Elements}
\begin{itemize}
\item Goals and tools are considered together
\item Means and ends are not separate
\item Values and facts are interconnected
\end{itemize}
\end{block}
\end{column}
\begin{column}{0.5\textwidth}
\begin{block}{Definition of ``Good'' Policy}
A ``good'' policy is one where \textcolor{titanorange}{\textbf{consensus can be reached}} among stakeholders.
\end{block}
\end{column}
\end{columns}
\end{frame}

% Satisficing in Bounded Rationality (reduced, no tikz)
\begin{frame}{Satisficing in Bounded Rationality}
\begin{block}{Key Principles}
\begin{itemize}
\item Administrative actors choose the first option that meets \textcolor{titanorange}{\textbf{minimum criteria}}
\item Makes the most rational decision with available information
\item Achieves satisfactory (not maximum) social gain
\item Recognizes that further search for solutions has costs
\end{itemize}
\end{block}
\end{frame}

% Example: Bounded Rationality
\begin{frame}{Bounded Rationality: Example}
\begin{block}{Case: City Homelessness Response}
\end{block}
\begin{columns}
\begin{column}{0.5\textwidth}
\textbf{Decision Context}
\begin{itemize}
\item Rising homeless population
\item Mayor facing re-election in 6 months
\item Limited city budget
\item Incomplete data on homeless demographics
\item Multiple stakeholders with competing interests
\end{itemize}
\end{column}
\begin{column}{0.5\textwidth}
\textbf{Satisficing Approach}
\begin{itemize}
\item Review 3-4 policy options (not all possible alternatives)
\item Set minimum criteria: implementable within 4 months, cost under \$2M, address immediate shelter needs
\item Select first option that meets all criteria
\item Choose temporary shelter expansion despite knowing it's not the optimal solution
\end{itemize}
\end{column}
\end{columns}
\end{frame}

% Incrementalism (reduced, no tikz)
\begin{frame}{Incrementalism}
\begin{block}{Foundation}
Builds on Herbert Simon's work on bounded rationality
\begin{itemize}
\item Recognizes limited information processing capacity
\item Focuses on making small, manageable changes
\item Reduces risk through incremental adjustments
\end{itemize}
\end{block}
\end{frame}

% Successive Limited Comparisons (no tikz)
\begin{frame}{Successive Limited Comparisons}
\begin{itemize}
\item Compare alternatives to the \textcolor{titanorange}{\textbf{status quo}}
\item Choose the alternative that is the \textcolor{titanorange}{\textbf{least different}} from current policy
\end{itemize}
\end{frame}

% Benefits of Incrementalism (no tikz)
\begin{frame}{Benefits of Incrementalism}
\begin{columns}
\begin{column}{0.5\textwidth}
\begin{block}{Simplifies Decision-Making}
\begin{itemize}
\item Reduces alternatives to consider
\item Focuses on marginal changes
\item Allows reliance on feedback
\end{itemize}
\end{block}
\end{column}
\begin{column}{0.5\textwidth}
\begin{block}{Manages Risk}
\begin{itemize}
\item Makes process serial and remedial
\item Avoids large, irreversible errors
\item Enables course correction
\end{itemize}
\end{block}
\end{column}
\end{columns}
\end{frame}

% Limitations of Incrementalism
\begin{frame}{Limitations of Incrementalism}
\begin{block}{Not Always Appropriate}
\begin{itemize}
\item Some problems are too complex for incremental solutions
\item Some problems are too urgent to address incrementally
\item Some problems require fundamental, not incremental, change
\end{itemize}
\end{block}

\textbf{When ``Muddling Through'' Won't Work}
\begin{itemize}
\item Moonshots \& major technological initiatives
\item Responses to wars \& national security threats
\item Managing pandemics \& public health emergencies
\item Addressing economic depressions \& major recessions
\end{itemize}
\end{frame}

% Incrementalism Example
\begin{frame}{Incrementalism: Example}
\begin{block}{Case: Environmental Regulation}
\end{block}
\begin{columns}
\begin{column}{0.5\textwidth}
\textbf{Status Quo}
\begin{itemize}
\item Current emissions standard: 30 parts per million
\item Industry has invested in existing compliance technology
\item Environmental groups want 10ppm standard
\item Economic concerns about rapid changes
\end{itemize}
\end{column}
\begin{column}{0.5\textwidth}
\textbf{Incremental Steps}
\begin{itemize}
\item Year 1: Reduce to 25ppm
\item Year 3: Reduce to 20ppm
\item Year 5: Reduce to 15ppm
\item Year 7: Consider further
\end{itemize}
\vspace{0.3cm}
\footnotesize{Each step builds on previous experience and allows for adjustment based on feedback and new information.}
\end{column}
\end{columns}
\end{frame}

% Three Models Compared (no tikz)
\begin{frame}{Three Models Compared}
\begin{block}{Summary Table}
\begin{tabularx}{\textwidth}{X|X|X}
\textbf{Rational Choice} & \textbf{Bounded Rationality} & \textbf{Incrementalism} \\
\hline
Complete info; Optimization; Comprehensive & Limited info; Satisficing; Consensus & Small changes; Status quo; Serial process \\
\end{tabularx}
\end{block}
\end{frame}

% When to Use Each Model (no tikz)
\begin{frame}{When to Use Each Model}
\begin{itemize}
\item \textbf{Rational Choice:} When information is complete and time allows for comprehensive analysis.
\item \textbf{Bounded Rationality:} When information is limited but consensus is needed.
\item \textbf{Incrementalism:} When only small changes are feasible or risk needs to be minimized.
\end{itemize}
\end{frame}

% Key Takeaways (keep, but remove tikz)
\begin{frame}{Key Takeaways: Alternative Models}
\begin{itemize}
\item \textbf{Bounded Rationality} acknowledges human cognitive limitations
\item \textbf{Satisficing} can be more practical than optimizing
\item \textbf{Incrementalism} reduces risk through small changes
\item Different situations call for different approaches
\end{itemize}
\end{frame}

% --- END SLIDES CONTENT ---

\end{document}
