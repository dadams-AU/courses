
\begin{frame}{Analyst Roles}

\tikzbox{
    \textbf{Objective Technician:}
    \begin{itemize}
        \item Independent, neutral
        \item Predicts outcomes, avoids advocacy
    \end{itemize}

    \textbf{Client's Advocate:}
    \begin{itemize}
        \item Supports client's position
        \item Strategic use of evidence
    \end{itemize}

    \textbf{Issue Advocate:}
    \begin{itemize}
        \item Uses analysis to push societal goals
        \item Accepts role as political actor
    \end{itemize}
}

\end{frame}


\begin{frame}{Bardach’s Eightfold Path}

\begin{tikzpicture}[node distance=1.2cm, every node/.style={rectangle, draw=titanblue, fill=skyblue!20, rounded corners, text width=0.9\textwidth, align=left, font=\small}]
\node (step1) {\textbf{1. Define the problem}};
\node (step2) [below=of step1] {\textbf{2. Assemble evidence}};
\node (step3) [below=of step2] {\textbf{3. Construct alternatives}};
\node (step4) [below=of step3] {\textbf{4. Select criteria}};
\node (step5) [below=of step4] {\textbf{5. Project outcomes}};
\node (step6) [below=of step5] {\textbf{6. Confront trade-offs}};
\node (step7) [below=of step6] {\textbf{7. Decide}};
\node (step8) [below=of step7] {\textbf{8. Tell your story}};
\end{tikzpicture}

\end{frame}


\begin{frame}{Two Logics of Policy}

\tikzbox{
    \textbf{Economic Rationality (Analysts):}
    \begin{itemize}
        \item Transparent assumptions
        \item Compare alternatives systematically
    \end{itemize}

    \textbf{Political Rationality (Policymakers):}
    \begin{itemize}
        \item Incentive-driven
        \item Selectively emphasize data
        \item Consider feasibility
    \end{itemize}
}

\end{frame}


\begin{frame}{Bridging the Logics}

\tikzbox{
    \begin{itemize}
        \item Both views are valid
        \item Policy isn't purely rational — or irrational
        \item Political realities matter
        \item Good analysis blends rigor with context
    \end{itemize}
}

\end{frame}


\begin{frame}{Why Politics Matters}

\tikzbox{
    \begin{itemize}
        \item Explains “irrational” outcomes
        \item Informs institutional design
        \item Prepares analysts to act strategically
        \item Values and preferences matter in policy
    \end{itemize}
}

\end{frame}


\begin{frame}{Evolution of the Profession}

\tikzbox{
    \begin{itemize}
        \item More diverse, specialized roles
        \item Analysts are not neutral technicians
        \item Engagement and inclusion are emphasized
        \item Technical skills still essential, but not enough
    \end{itemize}
}

\end{frame}


\begin{frame}{Let’s Try It: Wild Horse Case}

\tikzbox{
    \textbf{The Problem:} Wild horse overpopulation on public lands

    \textbf{Goals:} Sustainability, ecological health, humane treatment

    \textbf{Alternatives:}
    \begin{itemize}
        \item Adoption
        \item Fertility control
        \item Habitat expansion
    \end{itemize}

    \textbf{Criteria:} Cost, effectiveness, feasibility, animal welfare
}

\end{frame}


\begin{frame}{Video: Horse Rich and Dirt Poor}

\callout{Watch: \url{https://www.youtube.com/embed/q6h242vy_q8}}

\end{frame}


\begin{frame}{Reflections on the Case}

\tikzbox{
    \begin{itemize}
        \item Balance values: ecology, economy, ethics
        \item Mix of strategies may be required
        \item Must account for stakeholder perspectives
        \item Evaluation + adaptation are key to success
    \end{itemize}
}

\end{frame}


\begin{frame}{That’s All for Today}

\centering
\callout{Questions? Comments?}

\end{frame}
