% POSC 315 - Syllabus - Fall 2023

\documentclass[12pt, letterpaper]{article}
\usepackage[margin=.85in]{geometry}
\usepackage{xcolor}
\usepackage[colorlinks=true, linkcolor=blue, urlcolor=blue, citecolor=blue]{hyperref}
\usepackage{url}
\usepackage{graphicx}
\usepackage{bookmark}
\usepackage{enumitem}
\usepackage{caption}
\usepackage{longtable}
\captionsetup{font=large}
\captionsetup[table]{labelformat=empty} % Remove the "Table n" prefix
\usepackage{booktabs}
\frenchspacing
\usepackage{multicol}
\usepackage{eso-pic}
\usepackage{fontspec}
\setmainfont{Ubuntu}

\hypersetup{
  pdftitle={Introduction to Public Policy},
  pdfkeywords={315, syllabus, fall 2023},
  pdfcreator={Visual Studio Code Editor}
}
\begin{document}
\title{Introduction to Public Policy}
\author{POSC 315-01 -- Fall 2023}
\date{ Tuesdays and Thursdays at 11:30 in LH 401}

    \maketitle


\subsection*{Professor: David P. Adams, Ph.D.}

\subsubsection*{Contact Information:}

\begin{itemize}
	\item Office: 516 Gordon Hall
	\item Phone/SMS: (657) 278-4770
	\item website: \href{https://dadams.site}{\texttt{https://dadams.site}}
	\item email: \href{dpadams@fullerton.edu}{\texttt{dpadams@fullerton.edu}}
	\item Office hours: Tuesdays from 9:30 to 11:00 and by \href{https://t.ly/dpa-appt}{appointment}.
	\item Schedule meetings throughout the week: \href{https://t.ly/dpa-appt}{\texttt{https://t.ly/dpa-appt}}
\end{itemize}


\section*{Catalog Description}

	Federal domestic policymaking. Structure, functions, and relationships among American national institutions, including executive, legislative and judicial branches, media, political parties, and pressure groups.

\section*{Course Description}

	In this course, students will explore and engage in thoughtful discussions on the processes and key players in creating public policy in the United States. The curriculum focuses on the structure, functions, and relationships among American national institutions, including the executive, legislative, and judicial branches of government, the media, political parties, and interest groups. We will examine the various official and unofficial influences on the policy process and the limitations imposed by institutional and structural factors.

	This course delves into the historical and constitutional development of the policy process, as well as the distinct characteristics of public policy within a federal system. By understanding policy-making in the context of a constitutional republic and a federalized governance system, students will gain a deeper appreciation for the complexities and uncertainties surrounding agenda setting, policy making, policy implementation, and policy evaluation in the American political system.

\section*{Student Learning Objectives}

By the end of this course, students will be able to

\begin{enumerate}
	\item discuss and explain the key features of the public policy-making process in the United States;
	\item recognize and describe the distinct stages of the public policy process;
	\item describe the various internal and external actors that influence public policy, their interactions, and their impact on the policy process;
	\item articulate the historical and contemporary structures and institutions that facilitate, expand, or constrain the public policy process;
	\item differentiate and describe the various theories that attempt to explain the drivers and influences leading to public policy change or maintaining the status quo; and
	\item apply their knowledge of the policy process to analyze specific policy domains impacted by multiple policy actors and diverse elements of the policy process within the context of public policy-making in the United States.
\end{enumerate}

\section*{Text Book}

Birkland, Thomas A. 2020. \emph{An Introduction to the Policy Process: Theories, Concepts, and Models of Public Policy Making.} 5th edition. New York: Routledge.

\section*{General Education Information}

\subsection*{Requirements Satisfied}

	This course satisfies General Education Explorations in Social Sciences subarea D.5 for those using Catalog Years 2017 or earlier and subarea D.4 for those using Catalog Years 2018 and later. The writing assignments in this course, including the policy memo papers and current event summaries described below, meet the requirement of UPS 411.201: 
	\begin{quote}Writing assignments in General Education courses shall involve the organization and expression of complex data or ideas and careful and timely evaluations of writing so that deficiencies are identified, and suggestions for improvement and/or for means of remediation are offered. Evaluations of the student's writing competence shall determine the final course grade\ldots .\end{quote}

\subsection*{General Education Student Learning Goals}

	Students completing courses in this subarea shall encounter the following learning goals:

\begin{enumerate}
	\item Examine problems, issues, and themes in the social sciences in greater depth; in a variety of cultural, historical, and geographical contexts; and from different disciplinary and interdisciplinary perspectives.
	\item Analyze and critically evaluate the application of social science concepts and theories to particular historical, contemporary, and future problems or themes, such as economic and environmental sustainability, globalization, poverty, and social justice.
	\item Analyze and critically evaluate constructs of cultural differentiation, including ethnicity, gender, race, class, and sexual orientation, and their effects on the individual and society.
	\item Apply theories and concepts from the social sciences to address historical, contemporary, and future problems confronting communities at different geographical scales, from local to global.
\end{enumerate}

\section*{Student Policies}


\subsection*{University Policies}

	In accordance with UPS 300.00, students must be familiar with certain policies applicable to all courses. Please review these policies as needed and visit this \href{http://fdc.fullerton.edu/teaching/syllabus.php}{Cal State Fullerton website} (\href{https://t.ly/csuf-syllabus}{\texttt{https://t.ly/csuf-syllabus}}) for links to the following information:

\begin{enumerate}
	\item University learning goals and program learning outcomes.
	\item Learning objectives for each General Education (GE) category.
	\item Guidelines for appropriate online behavior (netiquette).
	\item Students' rights to accommodations for documented special needs.
	\item Campus student support measures, including Counseling \& Psychological Services, Title IV and Gender Equity, Diversity Initiatives and Resource Centers, and Basic Needs Services.
	\item Academic integrity (refer to UPS 300.021).
	\item Actions to take during an emergency.
	\item Library services information.
	\item Student Information Technology Services, including details on technical competencies and resources required for all students.
	\item Software privacy and accessibility statements.
\end{enumerate}

Refer to these policies and resources throughout the course as necessary.


\subsection*{Course Policies}

	\subsubsection*{Course Communication}
		All course announcements and communications will be sent via \emph{Canvas} and university email. Students are responsible for regularly checking their \emph{Canvas} notifications and email. Students are also responsible for ensuring that their \emph{Canvas} notifications are set to receive messages from the course. Students are expected to check \emph{Canvas} and their email at least once daily.
		
		\textbf{Response Time}: I will strive to respond to all student emails and \emph{Canvas} messages within 24 hours, except on weekends and holidays. If you do not receive a response within 24 hours, please send a follow-up message. If you do not receive a response within 48 hours, please send another follow-up message and contact me via phone or SMS text at (657) 278-4770.

	\subsubsection*{Due Dates}
		Please know that exams are only permitted on the scheduled date as indicated in the course schedule below. If you have concerns about meeting assignment deadlines, please contact the professor in advance to discuss potential accommodation.

	\subsubsection*{Alternative Procedures for Submitting Work}
		Students are expected to submit all assignments via \emph{Canvas}. If you cannot submit an assignment via \emph{Canvas}, please contact the professor to discuss alternative submission procedures.
	
	\subsubsection*{Extra Credit}
		There are no extra credit opportunities in this course. Please do not ask for extra credit assignments.

	\subsubsection*{Academic Integrity}
		Students are expected to adhere to the highest standards of academic integrity. Any student found to have engaged in academic dishonesty will be subject to the sanctions described in the \href{https://www.fullerton.edu/senate/publications_policies_resolutions/ups/UPS%20300/UPS%20300.021.pdf}{Academic Dishonesty Policy} (UPS 300.021). Academic dishonesty includes but is not limited to, cheating, plagiarism, fabrication, facilitating academic dishonesty, and submitting previously graded work without prior authorization. Students are expected to be familiar with the university's policy on academic dishonesty and to adhere to this policy in all aspects of this course. Any student who has questions about the policy should ask the professor for clarification.


\section*{Coursework, Evaluations, and Grades}

\subsection*{Three Examinations}

	Students will take three non-cumulative exams, each comprising 50 multiple-choice questions. These exams are designed to assess the understanding of key learning objectives, American politics, public policy formulation, and the policy-making process.

\subsection*{Three Documentary Response Papers}

	Students must complete three documentary response papers to explore and apply classroom concepts to real-world policy scenarios. Detailed guidelines will be provided on \emph{Canvas}. Each paper, approximately 1,000 words, should be submitted via the specified \emph{Canvas} assignment prompt.

\subsection*{Two Current Event Discussion Posts}

	Students must identify and analyze a current public policy event within the United States. A link or upload of the news story---from a reputable source---must be submitted to Canvas. Additionally, students will compose a concise paragraph about the event on the Canvas discussion board and comment on a peer's post. Students are expected to engage in class discussions about these events on designated days, linking them to the policy process. The Canvas submission is due the day before the class discussion.

\subsection*{Attendance and Participation}

	Regular attendance and active participation are vital for gaining the most from this course. Students are urged to participate in class dialogues concerning relevant policy subjects. Attendance will be recorded during each session. Except for emergencies, any deviations from the standard attendance schedule must be communicated to and approved by the professor in advance.


	
    \begin{center}
        \begin{minipage}[t]{.44\textwidth}
            \centering
            \captionof{table}{Course Grading Scheme}
            \begin{tabular}{l|c}
                \textbf{Assignment} & \textbf{Percentage} \\ \hline
            3 Exams & 60\% \\
            3 Documentary Papers & 30\% \\
            2 Current Event Summaries & 5\% \\
            Attendance & 5\% \\ \hline
            \textbf{Total} & \textbf{100\%} \\
            \end{tabular}
            \label{tab:grading_scheme}
        \end{minipage}\hspace{.07\textwidth}
        \begin{minipage}[t]{.44\textwidth}
            \centering
            \captionof{table}{Letters \& Percentages}
            \begin{tabular}{ll|ll}
                \textbf{Grade} & \textbf{Percentage} & \textbf{Grade} & \textbf{Percentage} \\ \hline
                A+ & 98.0 -- 100 & C+ & 78.0 -- 79.9 \\
                A  & 92.0 -- 97.9 & C  & 72.0 -- 77.9 \\
                A- & 90.0 -- 91.9 & C- & 70.0 -- 71.9 \\
                B+ & 88.0 -- 89.9 & D+ & 68.0 -- 69.9 \\
                B  & 82.0 -- 87.9 & D  & 62.0 -- 67.9 \\
                B- & 80.0 -- 81.9 & D- & 60.0 -- 61.9 \\
                    &               & F  & 0.0 -- 59.9 \\
                \hline
            \end{tabular}
            \label{tab:letter_grades}
        \end{minipage}
    \end{center}
\section*{Course Schedule}
\vspace{1cm}

    \begin{center}
\begin{longtable}{p{2cm} | p{6.8cm} | p{6.8cm}}	
	\label{tab:schedule} \\
	\large{\textbf{week}} & \large{\textbf{Tuesday}} 						& \large{\textbf{Thursday}} 						\\ \hline \hline
	
	\emph{week 1} 	& \underline{8/22:} \textbf{Introduction}				& \underline{8/24:} \textbf{Overview of Policy} 	\\
	\emph{Readings}	& Syllabus 												& \textit{Birkland ch. 1}							\\
					& 														&	American Regime Values							\\ \hline
	
	\emph{week 2} 	& \underline{8/29:} \textbf{Policy Process Elements}	&  \underline{8/31:} \textbf{Policy Process Elements} 	\\
	\emph{Readings} & \textit{Birkland ch. 2}								&  													\\
					& 														& 													\\ \hline
	
	\emph{week 3}	& 	\underline{9/5:} \textbf{History \& Structure}		& \underline{9/7:} \textbf{Documentary 1}			\\
	\emph{Readings}	& 	\textit{Birkland ch. 3}								& 													\\
					& 														& 													\\ \hline
	
	\emph{week 4}	& \underline{9/12:} \textbf{Official Policy Actors}		& \underline{9/14:} \textbf{Unofficial Policy Actors}		\\
	\emph{Readings}	& \textit{Birkland ch. 4 }								& \textit{Birkland ch. 5}							\\
					&														& 													\\ \hline
				
	\emph{week 5}	& \underline{9/19:} \textbf{Test 1 Review}		 		& \underline{9/21:} \textbf{TEST 1} 				\\
	\emph{Readings}	& 														& 													\\
					& \textbf{Documentary 1 Paper Due}						& 													\\ \hline	

	
	\emph{week 6}  	& \underline{9/26:} \textbf{Agenda Setting}				& \underline{9/28:} \textbf{Policy Types}			\\
	\emph{Readings}	& \textit{Birkland ch. 6}								& \textit{Birkland ch. 7} 							\\
					& 														& 													\\ \hline	
	
	\emph{week 7}	& \underline{10/3:} \textbf{Decision Making}			& \underline{10/5:} \textbf{Current Event Day 1}	\\
	\emph{Readings}	& \textit{Birkland ch. 8}								& 												 	\\
					& 														& 													\\ \hline	
	
	\emph{week 8}	& \underline{10/10:} \textbf{Documentary 2} 			& \underline{10/12:} \textbf{Documentary 2}			\\
	\emph{Readings}	& \textit{No in-person class}							& \textit{No in-person class}						\\
					& 														& 													\\ \hline	
	
	\emph{week 9}	& \underline{10/17:} \textbf{Policy Analysis}			& \underline{10/19:} \textbf\textbf{Test 2 Review}	\\
	\emph{Readings}	& \textit{Birkland ch. 8}								& 	  												\\
					& 														& \textbf{Documentary 2 Paper Due}					\\ \hline	
	
	\emph{week 10} 	& \underline{10/24:} \textbf{TEST 2}					& \underline{10/26:} \textbf{Current Event Day} 	\\
	\emph{Readings}	& 														& \textit{Let's just chat about the world}			\\ 
					&														&													\\\hline	
	
	\emph{week 11}  & \underline{10/31:} \textbf{Policy Design \& Tools}	& \underline{11/2:} \textbf{Implementation}			\\
	\emph{Readings}	& \textit{Birkland ch. 9}								& \textit{Birkland ch. 10}							\\
					& 														& 													\\ \hline	
	
	\emph{week 12} 	& \underline{11/7:} \textbf{Failure and Learning} 		& \underline{11/9:} \textbf{Documentary 3}			\\
	\emph{Readings}	& \textit{Birkland ch. 10}								& \textit{Birkland ch. 11}							\\
					&														& 													\\\hline	
		
	\emph{week 13} 	& \underline{11/14:} \textbf{Evaluation \& Science}		& \underline{11/16:} \textbf{Documentary Paper 3 Due}	\\
	\emph{Readings}	& \textit{Birkland ch. 11}								& 													\\
					& 														& \textit{No In-Person Class}						\\ \hline	
	
	\emph{week 14} 	& \underline{11/28:} \textbf{Policy Sciences}	 		& \underline{11/30:} \textbf{Current Event Day 2}	\\
	\emph{Readings}	& \textit{Birkland ch. 11}								& 													\\
					& 														& 													\\ \hline	
	
	\emph{week 15}	& \underline{12/5:} \textbf{Collaboration}				& \underline{12/7:} \textbf{Final Exam Review}		\\
	\emph{Readings}	& \textit{Canvas Reading} 								& 					 								\\
					& Wrap-up remaining topics								&													\\ \hline	
					
	\emph{week 16} &														& \underline{12/14:} \textbf{TEST 3}				\\
				   &														& 11:00 \textsc{a.m.} -- 12:50 \textsc{p.m.}		\\
				   &														& LH 401											\\ \hline
					
	
\end{longtable}
\end{center}

\section*{Important Scheduling Note}

Please be aware the California Faculty Association---the labor union of Lecturers, Professors, Coaches, Counselors, and Librarians across the 23 CSU campuses---is navigating challenging contract negotiations with CSU management, and a strike or work stoppage may occur this term. Our working conditions are your learning conditions; we seek to protect both. For updates, visit \href{www.CFAbargaining.org}{\texttt{www.CFAbargaining.org}}.

\end{document}

\end{document}
